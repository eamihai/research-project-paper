\section{Discussion}
\label{sec: discussion}
Our experiments show that the PPS workload complements the commonly used workloads TPC-C and YCSB+T by covering several gaps they leave open, but it doesn't address every evaluation need. In this section, we outline the main advantages of using the PPS workload (Section~\ref{subsec: advantages-of-the-pps-workload}) and also acknowledge its limitations (Section~\ref{subsec: limitations-of-the-pps-workload}).

\subsection{Advantages of the PPS Workload}
\label{subsec: advantages-of-the-pps-workload}
\paragraph{Dependent Transactions.}
PPS models many-to-many relationships between products, parts, and suppliers. Thus, transactions need to work with \textit{foreign key dependencies} when joining different tables, which is not the case in the single-table YCSB+T workload and the TPC-C warehouse hierarchy. In particular, the \textit{OrderProduct} type needs to perform a read phase to discover all the relevant parts before being able to perform the actual updates. If another client changes the list of parts in between, the transaction aborts and retries. Neither TPC-C nor YCSB+T ever forces the systems to abort, so PPS is the only one of the three workloads that stresses this case. We captured how increasing the skew factor impacts the resulting abort rates in Section~\ref{subsubsec: skew-access-scenario}.

\paragraph{Longer Transaction Footprint.}
PPS allows us to choose the number of parts per product, and thus control the size of the \textit{read/write sets} of both \textit{GetPartsByProduct} and \textit{OrderProduct} transactions. In our experiments, we set this parameter to 10. The larger access sets in PPS particularly penalize the protocols whose cross-region coordination grows in complexity as the number of keys increases. Janus is an example of such a design, where a larger access set would lead to bigger messages and more potential conflicts, which pushes the transaction on the slow path that needs a second WAN round-trip. As a consequence, Janus's throughput relative to the other systems in our PPS baseline (Section~\ref{subsubsec: baseline-scenario}) is far lower than in Nguyen et al.'s study \cite{nguyen2023detock}, which used YCSB+T.

\paragraph{Fine-grained Home Region Control.}
Besides the ability to dynamically generate multi-home or single-home transactions, which is an option already discussed in the Detock performance evaluation \cite{nguyen2023detock}, our PPS-based benchmarking framework allows us to redirect a given fraction of transactions to a particular region. For a single-home transaction, we can select the exact region that will coordinate it, and for a multi-home transaction, we can pick a specific region to appear in the home set. This flexibility was essential in running the sunflower experiment (Section~\ref{subsubsec: sunflower-topology-scenario}), where we gradually shifted the traffic to a region to create a realistic \textit{hotspot}.

\subsection{Limitations of the PPS Workload}
\label{subsec: limitations-of-the-pps-workload}
\paragraph{Limited Updates.}
Unfortunately, only two transactions in the PPS workload mix perform writes, namely \textit{OrderProduct} and \textit{UpdateProductPart}. Each of them modifies records from a \textit{single table}. On the other hand, TPC-C offers a broader spectrum of write patterns, which can lead to more interesting contention challenges that PPS cannot reproduce.

\paragraph{Multi-home and Multi-partition Limitations.}
Because only the \textit{OrderProduct} transaction can be configured to be multi-home or multi-partition, the workload mix must be adjusted to consider this. Even though we can choose the percentages of \textit{OrderProduct} transactions in the mix, it is still a dependent transaction whose read-only first phase is always single-home and single-partition, and thus we cannot achieve an \textit{overall combination} where $100\%$ of the generated transactions are multi-home and multi-partition. On the other hand, even though YCSB+T is simpler, it allows arbitrary key distributions, so it can be used to reach any desired proportion of multi-home and multi-partition transactions.