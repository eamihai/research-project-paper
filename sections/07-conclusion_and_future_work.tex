\section{Conclusions and Future Work}
\label{sec: conclusion-and-future-work}

This work introduced a configurable benchmarking framework based on the PPS workload and demonstrated how it can be used to augment the standard workloads TPC-C and YCSB+T. Our benchmark adds dependent transactions, larger read/write sets, and fine-grained control over single-home and multi-home access patterns, which are key aspects when evaluating any geo-distributed database system. 

Using this benchmark, we compared four such systems, Calvin, SLOG, Detock, and Janus, across six different scenarios. The results show clear tradeoffs. Calvin's global sequencer pays off once there's a majority of transactions that access multiple regions, whereas SLOG and Detock excel when most transactions can be executed locally. Janus's bottleneck is the required WAN round-trip approval, and the larger PPS access set can potentially lead to another WAN round-trip to solve the conflicts, which generally proved to have low performance in our two-region deployment. The skew experiment showed the hidden cost of having dependent transactions through increasing abort rates, while the sunflower experiment exposed how systems that are optimized based on locality can overload one replica.

Ultimately, the purpose of this paper is to raise awareness of the need for benchmarks suited for geo-distributed databases. Although PPS fills some gaps, it has its limitations. Future extensions could introduce additional transactions that are write-heavy or have the potential to span multiple regions. Moreover, it might be worth exploring how the results would be affected by having server-side reconnaissance (OLLP) for dependent transactions.